\documentclass{article}
\usepackage[utf8]{inputenc}

\title{%
  I Giochi del Fede\\
  \large Ed ulteriori stronzate}

\author{I Reggiani}
\date{September 2021}

\begin{document}

\maketitle
\newpage
\tableofcontents
\newpage
\section{Introduzione}
Codesto libro viene scritto in un clima di generale euforia post-esami, misto alla (talvolta) malsana spensieratezza caratteristica dello studente universitario medio.

Lo scopo principale e' quello di descrivere nel dettaglio i passatempi di uno specifico studente dell'Università degli Studi di Trento, che risponde al nome di Federico (da qui in poi: “Il Creatore”). Il cognome verrà omesso per evitare che la futura notorietà che accoglierà questo libro possa in alcun modo influenzare la tranquilla vita quotidiana del sopracitato.

Tali passatempi potrebbero venir giudicati dall'occhio meno attento come “giochini infantili”, o peggio ancora come “intrattenimenti bambineschi”.

Solo chi sarà in grado di aprire la mente ad una visione d'insieme potrà apprezzare al meglio le attività qui sotto proposte: i lettori più attenti riusciranno a farsi cullare tra le corde di un'amaca leggiadra, spinta da un lato dalla fanciullesca innocenza Pascoliana, e dall'altro  dolcemente accompagnata dagli intrattenenti motivetti del Creatore.

Le attività che verranno descritte a breve non sono da intendersi come imposizioni di un ente esterno, bensì possono funzionare solo se provenienti da un bisogno interno, da una tendenza al rivivere i momenti più intimi del nostro passato.

Non si pensi che questo libro sia una semplice e sciatta raccolta di passatempi del Creatore; e' invece stata effettuata un'attenta selezione, non priva di buon gusto, affinché' il lettore si concentrasse sugli aspetti principali piuttosto che su dettagli fuorvianti, e riuscisse ad entrare nell'ottica corretta per un godimento adeguato.

E' consigliata la lettura sequenziale, nonostante, date le proprietà intrinsecamente descrittive di questo libro, non ci sia una trama od un filo logico, ma sia piuttosto caratterizzato da un percorso di crescita e maturazione.

Gli autori sono tuttavia disposti ad accettare anche una lettura disomogenea e disordinata, soprattutto se persiste l'intento di una “consultazione” del libro visto come manuale, con lo scopo di colmare una eventuale dimenticanza.

E' quindi suggerito l'abbandono di ogni tipo di pregiudizio, in quanto ne' il Creatore ne' i collaboratori saranno interessati ad eventuali lamentele, ed il lettore si troverebbe quindi a dolersi solamente con se' stesso.

Anche le critiche costruttive sono scoraggiate, siccome verranno probabilmente mal accolte da un “ ormai l'abbiamo fatto cosi' ”, ma soprattutto dato che l'orgoglio de “I Reggiani” e' sacro.

Chiunque riuscirà in questo intento potrà adagiarsi tra le braccia del Creatore, e lasciarsi cantare una canzone antica, apprezzando tuttavia un'ideologia e uno stile di vita del tutto nuovi.

\newpage
\section{Giochi riguardo l'automobile}

\subsection{Imminenza}

\newpage
\section{Giochi della vita quotidiana}
\subsection{Mossa}
Sin dai tempi antichi, quando l'unica cosa importante era sopravvivere, il genere umano soleva stuzzicarsi l'umorismo con un semplice piccolo giochetto, che riusciva a spezzare anche il più duro momento di imbarazzo o tempo morto.
Questo deriva dallo sfrenato feticcio di infilare parti del corpo in buchi di qualsiasi genere, che accompagna l'uomo da prima della completa evoluzione in Homo Sapiens.
Detto ciò spostiamoci sul gioco vero e proprio: consiste nel porre la mano con indice e pollice uniti per le punte al di sotto della sporgenza situata sotto la cavità orale, comunemente chiamata mento.
Se quest'ultimo dettaglio venisse tralasciato l'azione non avrebbe effetto sul bersaglio, anzi, l'aggressore riceverebbe vergogna in egual quantità rispetto a quella causata dal gioco stesso nella vittima.
\newpage
\section{Conclusioni}

\newpage
\section{BabyBorlo}

\end{document}

